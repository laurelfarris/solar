\documentclass[preprint2]{aastex}
\usepackage{natbib}
\usepackage{color}
\usepackage{booktabs}
\bibliographystyle{apj}

\shorttitle{Bright Point Size}
\shortauthors{Farris}

\begin{document}

\title{Determining coronal bright point size via cross-correlation using
multi-wavelength images from AIA/\textit{SDO}}
\author{Laurel Farris, R. T. James McAteer}
\affil{New Mexico State University}
\email{laurel07@nmsu.edu}

\begin{abstract}
\end{abstract}
\keywords{Sun: corona{-}Sun: bright points{-}}

\section{Introduction}\label{intro}
It is currently thought that after magnetic flux tubes rise from the tachocline
of the sun (between the convection and radiative zones) to the photosphere, they
are moved across the photosphere by the process of advection to the junctions
between supergranules (source?). There they are observed as bright points (BPs).
Though they only cover about 1.6 \% of the
photosphere (\cite{Srivastava}), BPs (together with sunspots)
contribute over 90\% of the total magnetic flux (\cite{Howard}).
Over the course of the solar cycle, they can contribute significantly to the
global intensity variation of the sun, particularly in the ultraviolet
regime (\cite{Riethmuller}).

These BPs can be seen in the upper layers of the solar atmosphere in the form
of coronal BPs. The cross-sectional area of these BPs is known to increase with
height above the photosphere as the density decreases and temperature increases
(source?).

Several techniques for determining the size of coronal BPs have been investigated
in the literature.
\cite{Alipour} developed an algorithm to locate BPs in the corona, using size determined
by intensity as part of the criteria for distinguishing BPs from other features,
such as top-down views of coronal loops or nanoflares.

The goal for the project discussed in this Letter was to determine size in another
way, using the cross-correlation between the pixels in and around the visible
area occupied by the BP. The data is described in \S{} \ref{data},
the analysis is discussed in \S{} \ref{analysis},
and conclusions are in \S{} \ref{conclusion}.


\section{Data}\label{data}
This analysis was carried out using multi-wavelength data from AIA/\textit{SDO}
spanning one hour on June 1, 2012 from 13:00:00 to 13:59:59, at a cadence of 12
seconds.
Each EUV passband corresponds to emission
from different transitions of ion species in the corona, each of which takes
place at different temperatures, and hence different heights, above the
photosphere.
The relevant values for each passband are given in table \ref{temps}.
\begin{table}[h]
\centering
    \begin{tabular}{l l}
        \hline\hline
        Wavelength [\AA{}] & Temperature [K] \\
        \hline
        94 & $10^{6.8}$ \\
        131 & $10^{5.6}, 10^{7.0}$ \\
        171 & $10^{5.8}$ \\
        193 & $10^{6.2}, 10^{7.3}$ \\
        211 & $10^{6.3}$ \\
        304 & $10^{4.7}$ \\
        335 & $10^{6.4}$ \\
    \end{tabular}
\caption{Characteristic temperatures corresponding to the wavelengths observed
    in emission in the solar corona (taken from \cite{Lemen}).}
\label{temps}
\end{table}

%A grayscale image of the full disk at the beginning of the time series is shown in figure \ref{full}.
A single BP from a coronal hole located in the full disk was selected for analysis.
Grayscale images showing this BP in each passband are shown in figure \ref{full}.

\begin{figure*}[htb!]
    \includegraphics[width=\textwidth]{Figures/images.png}
    \caption{Images of the BP in six different AIA wavelengths.}
    \label{full}
\end{figure*}

\section{Analysis}\label{analysis}
The intensity of each BP as a function of radius gives a visual estimate
of the size of the BP. For comparison, the intensity of the first image in each
passband is plotted as a function of radial distance from the center pixel in figures
\ref{intensity_1} and \ref{intensity_2}.

For this project, the size was estimated by running a cross-correlation between
a pixel roughly in the center of the BP and the remaining pixels over the
100 $\times$ 100 pixel$^{2}$ area through the full hour long time series.
The central pixel was arbitrarily determined by locating the brightest pixel in the center
of the BP from the first image in the time series. The cross-correlation analysis
helps to determine what parts of the feature are moving together as a single
physical structure.

\section{Results}\label{results}
Images illustrating the highest cross-correlation value of each pixel and the
timelag corresponding to that correlation value are shown in figures \ref{cc_all}
and \ref{tt_all}. The correlation was cut off at 0.5 and rescaled to obtain a
better illustration of the structure of the BP. These images are shown in
figures \ref{cc} and \ref{tt}.

As \cite{Alipour} noted, the BP structure is most evident for the
131\AA{}, 193\AA{}, and 211\AA{} images.

The 211\AA{} data is particularly notable as it shows strong correlation values
at two points to the right of where the BP appears visually in figure \ref{full}.
A movie showing all images from this wavelength (\cite{ssw}) revealed a flash
of emission around the ?th image.
%which matches the timelag at which the high correlation values occurred.

\begin{figure*}[htb!]
    \includegraphics[width=\textwidth]{Figures/cc_all.png}
    \caption{Images showing the highest cross-correlation value for each pixel. }
    \label{cc_all}
\end{figure*}

\begin{figure*}[htb!]
    \includegraphics[width=\textwidth]{Figures/tt_all.png}
    \caption{Images showing the timelag corresponding to the correlation values
        illustrated in figure \ref{cc_all}. (This is not the correct image; I accidently
        overwrote the correct one, but can't re-create it because IDL is giving me
        syntax errors on my function keywords for no apparent reason.)}
    \label{tt_all}
\end{figure*}

\begin{figure*}[htb!]
    \includegraphics[width=\textwidth]{Figures/cc_images.png}
    \caption{Cross-correlation images scaled to show only values higher than 0.5.}
    \label{cc}
\end{figure*}

\begin{figure*}[htb!]
    \includegraphics[width=\textwidth]{Figures/tt_images.png}
    \caption{Timelag corresponding to the cross-correlation values higher than 0.5.}
    \label{tt}
\end{figure*}

\begin{figure*}[htb!]
    \includegraphics[width=\textwidth]{Figures/intensity_1.png}
    \caption{Intensity of each pixel is plotted as a function of radius for each
        passband.}
    \label{intensity_1}
\end{figure*}

\begin{figure*}[htb!]
    \includegraphics[width=\textwidth]{Figures/intensity_2.png}
    \caption{Same as figure \ref{intensity_1}, but with half the radius range
        cut off to better view the values around the main BP.}
    \label{intensity_2}
\end{figure*}

\begin{figure*}[htb!]
    \includegraphics[width=\textwidth]{Figures/cc_tt_plot.png}
    \caption{The highest cross-correlation value of each pixel is plotted as a function
        of its distance from the center pixel. The color indicates the timelag
        corresponding to the maximum cross-correlation for that pixel.}
    \label{tt_all_plot}
\end{figure*}

\begin{figure*}[htb!]
    \includegraphics[width=\textwidth]{Figures/cc_tt_plot_scaled.png}
    \caption{Same as figure \ref{tt_all_plot}, but with two thirds of the timelag cut out
        at both ends.}
    \label{tt_plot}
\end{figure*}


\section{Conclusion}\label{conclusion}

\bibliography{reffile}
\end{document}
