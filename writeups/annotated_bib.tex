\documentclass{article}
\renewcommand\familydefault{\sfdefault}
\usepackage[margin=1.75in]{geometry}
\setlength{\parindent}{0ex}
\setlength{\parskip}{0.5ex}
\usepackage{xcolor}
\usepackage{natbib}
%\bibliographystyle{apj}

\usepackage{titlesec}
%\setcounter{secnumdepth}{0}
\titlespacing*{\section}{-0.25in}{0ex}{0ex}
\titlespacing*{\subsection}{0in}{0ex}{0ex}

\usepackage{enumitem}
%\setlist[itemize]{%
\setitemize{%
    label={\footnotesize$\bullet$}
}
\setdescription{%
    style=nextline,
    itemindent=0em,
    leftmargin=2em,
    }
\renewcommand{\descriptionlabel}{%
    \bfseries}

\usepackage{hyperref}
\definecolor{cobalt}{rgb}{0.0, 0.28, 0.67}
\hypersetup{colorlinks=true, urlcolor=cobalt, linkcolor=black}
\urlstyle{same}

\begin{document}
\section{Papers}
\subsection{Granular structure}
\begin{description}
    \item [\href{http://cdsads.u-strasbg.fr/abs/1997ApJ...481..988H}
        {The distribution of cell sizes of the Solar Chromospheric Network}]
        from Priest, page 22, ``basin-finding'' algorithm for finding
        supergranules.
    \item [\href{http://cdsads.u-strasbg.fr/abs/2008A\%26A...479L..17R}
        {Solar supergranulation revealed by granule tracking}]
        Priest, page 22, granule tracking.
    \item [\href{http://cdsads.u-strasbg.fr/abs/1989ApJ...336..475T}
        {Statistical properties of solar granulation derived
        from the SOUP instrument on Spacelab 2}]
        Cited by Priest, having something to do with the motions of granules and supergranules.
    \item [\href{http://cdsads.u-strasbg.fr/abs/2000SoPh..193..313S}
        {Supergranule and mesogranule evolution}]
        Cited by Priest, along with November when
        discussing the difficulties of observing mesogranulation.
    \item [\href{http://adsabs.harvard.edu/cgi-bin/bib_query?arXiv:0902.2299}
        {Mesoscale dynamics on the Sun's surface from HINODE
        observations}]
    \item [\href{http://adsabs.harvard.edu/abs/1981ApJ...245L.123N}
        {The detection of mesogranulation on the sun}]
        the first to detect structure between granule
        and supergranule size scales.
\end{description}

\subsection{Alfv\'en waves}
\begin{description}
    \item [\href{http://arxiv.org/abs/0903.3546}
        {Alfv\'en waves in the lower solar atmosphere}
        - Jess, 2009]
    \item [\href{https://arxiv.org/abs/0910.0962}
        {The role of torsional Alfv\'en waves in coronal heating}
        - P. Antolin, K. Shibata]
\end{description}

\subsection{Instrumentation}
\begin{description}
    \item [\href{http://cdsads.u-strasbg.fr/abs/2012SoPh..275...17L}
            {The (AIA) on (SDO)}]
            Obviously$\ldots$ AIA info.
\end{description}

\subsection{Coronal bright points}
\begin{description}
    \item [\href{http://adsabs.harvard.edu/abs/2015ApJ...807..175A}
        {Statistical properties of solar coronal bright points}
        -Alipour \& Safari]
\end{description}

\subsection{Coronal seismology}
\begin{description}
    \item [\href{http://link.springer.com/article/10.1007\%2Fs11207-007-9029-z}
        {Present and Future Observing Trends in Atmospheric Magnetoseismology}]
    \item [\href{http://arxiv.org/abs/1304.5439}
        {Modeling the Line-of-Sight Integrated Emission in the Corona:
        Implications for Coronal Heating}
        - Viall and Klimchuk]
    \item [\href{http://rsta.royalsocietypublishing.org/content/370/1970/3193}
        {Magnetohydrodynamic waves and coronal seismology: an overview of recent results}]
        - Ineke De Moortel, Valery M. Nakariakov
    \item [\href{http://arxiv.org/abs/1509.05519}
        {Decayless low-amplitude kink oscillations: a common phenomenon in the solar corona?}]
    \item [\href{http://adsabs.harvard.edu/abs/2016A\%26A...585L...6P}
        {Damping profile of standing kink oscillations observed by SDO/AIA}]
\end{description}

\subsection{Other}
\begin{description}
    \item [\href{https://arxiv.org/abs/1208.4693}
        {Solar Force-free magnetic fields}]
        - Thomas Weigelmann
    \item [\href{http://cdsads.u-strasbg.fr/abs/1964ApJ...140.1120S}
        {Velocity fields in the solar atmosphere. III.
        Large-Scale Motions, the Chromospheric Network, and Magnetic Fields}]
        - Priest
        page 22, autocorrelation method for finding mean size of supergranules.
\end{description}

\section{Other links}
\begin{itemize}
    \item \url{http://solarphysics.livingreviews.org/open?pubNo=lrsp-2010-2&amp;page=articlesu5.html}
    \item \url{http://solarphysics.livingreviews.org/Articles/lrsp-2012-5/download/lrsp-2012-5Color.pdf}
    \item \url{http://dkist.nso.edu}
\end{itemize}

%\bibliography{reffile}
%\end{flushleft}
\end{document}
