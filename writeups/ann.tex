\documentclass{article}
\usepackage[margin=1in]{geometry}
\usepackage{hyperref}
\usepackage{color}
\usepackage{natbib}
%\bibliographystyle{apj}

\usepackage{titlesec}
\titleformat*{\section}{\large\normalfont}
\setcounter{secnumdepth}{0}

\usepackage{hyperref}
\definecolor{cobalt}{rgb}{0.0, 0.28, 0.67}
\hypersetup{colorlinks=true,
    urlcolor=cobalt,
    linkcolor=black
}
\urlstyle{same}

\usepackage{enumitem}
\setlist[itemize]{%
    itemsep=-1ex}

\begin{document}

\section{%
    \href{http://arxiv.org/abs/1304.5439}
    {Modeling the Line-of-Sight Integrated Emission in the Corona:
    Implications for Coronal Heating}
    - Viall and Klimchuk}
\section{%
    \href{http://arxiv.org/abs/0903.3546}
    {Alfv\'en waves in the lower solar atmosphere}
    - Jess, 2009}

\section{%
    \href{https://arxiv.org/abs/1208.4693}
    {Solar Force-free magnetic fields}
    - Thomas Weigelmann
}
\section{%
    \href{https://arxiv.org/abs/0910.0962}
    {The role of torsional Alfv\'en waves in coronal heating}}
    P. Antolin, K. Shibata

\section{%
\href{http://link.springer.com/article/10.1007\%2Fs11207-007-9029-z}
{Present and Future Observing Trends in Atmospheric Magnetoseismology}}

\section{%
\href{http://rsta.royalsocietypublishing.org/content/370/1970/3193}
{Magnetohydrodynamic waves and coronal seismology: an overview of recent results}}
Ineke De Moortel, Valery M. Nakariakov

\section{%
\href{http://arxiv.org/abs/1509.05519}
{Decayless low-amplitude kink oscillations: a common phenomenon in the solar corona?}}

\section{%
\href{http://adsabs.harvard.edu/abs/2016A\%26A...585L...6P}
{Damping profile of standing kink oscillations observed by SDO/AIA}
}

\section{%
\href{http://cdsads.u-strasbg.fr/abs/2012SoPh..275...17L}
{The (AIA) on (SDO)}}
Obviously$\ldots$ AIA info.

\section{%
\href{http://adsabs.harvard.edu/abs/1981ApJ...245L.123N}
{The detection of mesogranulation on the sun}}
the first to detect structure between granule
and supergranule size scales.

\section{%
Magnetohydrodynamics of the Sun}
Article review type book. Chapter 1, section 4 has some useful
information on granules, mesogranules, and supergranules.
Probably wouldn't cite the book in a paper; use the papers referenced
instead.

\section{%
\href{http://adsabs.harvard.edu/cgi-bin/bib_query?arXiv:0902.2299}
{Mesoscale dynamics on the Sun's surface from HINODE
observations}}

\section{%
\href{http://cdsads.u-strasbg.fr/abs/1989ApJ...336..475T}
{Statistical properties of solar granulation derived
from the SOUP instrument on Spacelab 2}}
Cited by Priest, having something to do with
the motions of granules and supergranules.

\section{%
\href{http://cdsads.u-strasbg.fr/abs/2000SoPh..193..313S}
{Supergranule and mesogranule evolution}}
Cited by Priest, along with November when
discussing the difficulties of observing mesogranulation.

\section{%
\href{http://cdsads.u-strasbg.fr/abs/1964ApJ...140.1120S}
{Velocity fields in the solar atmosphere. III.
Large-Scale Motions, the Chromospheric Network, and Magnetic Fields}}
Priest, page 22, autocorrelation method for finding mean size of
supergranules.

\section{%
\href{http://cdsads.u-strasbg.fr/abs/1997ApJ...481..988H}
{The distribution of cell sizes of the Solar Chromospheric Network}}
from Priest, page 22, ``basin-finding'' algorithm for finding
supergranules.

\section{%
\href{http://cdsads.u-strasbg.fr/abs/2008A\%26A...479L..17R}
{Solar supergranulation revealed by granule tracking}}
Priest, page 22, granule tracking.

\section{Other links}
\begin{itemize}
    \item \url{http://solarphysics.livingreviews.org/open?pubNo=lrsp-2010-2&amp;page=articlesu5.html}
    \item \url{http://solarphysics.livingreviews.org/Articles/lrsp-2012-5/download/lrsp-2012-5Color.pdf}
    \item \url{http://dkist.nso.edu}
\end{itemize}

%\bibliography{reffile}
\end{document}
