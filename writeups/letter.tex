\documentclass[preprint2]{aastex}
\usepackage{natbib}
\usepackage{color}
\usepackage{booktabs}
\bibliographystyle{apj}

\shorttitle{Bright Point Size}
\shortauthors{Farris}

\begin{document}

\title{Determining coronal bright point size via cross-correlation using
multi-wavelength images from AIA/\textit{SDO}}
\author{Laurel Farris, R. T. James McAteer}
\affil{New Mexico State University}
\email{laurel07@nmsu.edu}

\begin{abstract}
\end{abstract}
\keywords{Sun: corona{-}Sun: bright points{-}}

\section{Introduction}\label{intro}
Bright points in the junctions between supergranules in the solar photosphere
can be seen in the upper layers of the solar atmosphere in the form of coronal
bright points. The cross-sectional area of these BPs is known to increase in
height as the density decreases and temperature increases (source).
\section{Data}\label{data}
\section{Analysis}\label{analysis}
\section{Results}\label{results}
\section{Conclusion}\label{conclusion}

\bibliography{reffile}
\end{document}
