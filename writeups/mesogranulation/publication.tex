\documentclass[preprint2]{aastex}
\usepackage{color}
%\received{January 28 2016}
%\accepted{January 28 2016}

\begin{document}

\title{Here is my title}
\author{Laurel Farris}
\affil{New Mexico State University}
\email{laurel@laurelfarris.com}

\begin{abstract}
Here is my abstract.
\end{abstract}

\section{Introduction}
The structures of granulation and supergranulation have been well
established. The depths at which these cells penetrate into the
interior of the sun is correlated with the depth at which
hydrogen and helium are ionized. SG is connected with ionized
hydrogen, HII
(totally making this up since I forgot how this actually works),
granulation is connected with doubly ionized helium, HeIII.
However, while G and SG are evident across the solar surface, and
have been experimentally connected with interior phenomenae,
no intermediate cells
have been detected that would be connected with singly ionized helium,
HeII.
In 1981, November et al. initiated the search for mesogranulation.


\section{Data}
Data from the AIA instrument on SDO was used.
The bandpass centered on Fe ??? 191\AA{} was chosen because reasons.
Using the \emph{Helioviewer} 
website \textcolor{red}{SOURCE!} to find a `quiet' day (few sunspots or flares),
about an hour's worth of data was downloaded, at full cadence (12
seconds between each image). Since granulation occurs on times scales
of 10-12 minutes, and supergranulation occurs on times scales of about
2 days, mesogranulation is expected to occur on time scales somewhere
between those two. 

Each image was reduced from the original full disk 
4096$\times$4096 pixels to
about 2001$\times$2001 to cut out the limb.
The images were aligned to correct for pixel size shifts from one to
the next, using the middle image as the reference.
The end product was an extracted data-time series.
A cross-correlation was run on the data cube to obtain both the
timelag between points of high correlation, and the distance between
them.

\section{Results}
Figures will be shown with different parameters (variables) under
consideration: Intensity, distance from central pixel,
direction from central pixel and time of propagation (timelag).

\section{Conclusion}
And we're finished.

\acknowledgments I would like to thank people.




\end{document}







